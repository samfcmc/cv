%%%%%%%%%%%%%%%%%%%%%%%%%%%%%%%%%%%%%%%
% Wenneker Resume/CV
% LaTeX Template
% Version 1.1 (19/6/2016)
%
% This template has been downloaded from:
% http://www.LaTeXTemplates.com
%
% Original author:
% Frits Wenneker (http://www.howtotex.com) with extensive modifications by 
% Vel (vel@LaTeXTemplates.com)
%
% License:
% CC BY-NC-SA 3.0 (http://creativecommons.org/licenses/by-nc-sa/3.0/
%
%%%%%%%%%%%%%%%%%%%%%%%%%%%%%%%%%%%%%%

%----------------------------------------------------------------------------------------
%	PACKAGES AND OTHER DOCUMENT CONFIGURATIONS
%----------------------------------------------------------------------------------------

\documentclass[a4paper,12pt]{memoir} % Font and paper size

\input{structure.tex} % Include the file specifying document layout and packages

%----------------------------------------------------------------------------------------
%	NAME AND CONTACT INFORMATION 
%----------------------------------------------------------------------------------------

\userinformation{ % Set the content that goes into the sidebar of each page
\begin{flushright}
% Comment out this figure block if you don't want a photo
% \includegraphics[width=0.6\columnwidth]{photo.jpg}\\[\baselineskip] % Your photo
\small % Smaller font size
Samuel Coelho \\ % Your name
\url{samuelfcmc@gmail.com} \\ % Your email address
% \url{www.johnsmith.com} \\ % Your URL
(+351) 915010563 \\ % Your phone number
\Sep % Some whitespace
\textbf{Address} \\
Rua Esteiro da Quebrada 19 3º Esq\\ % Address 1
2870-420 Montijo \\ % Address 2
Portugal \\ % Address 3
\vfill % Whitespace under this block to push it up under the photo
\end{flushright}
}

%----------------------------------------------------------------------------------------

\begin{document}

\userinformation % Print your information in the left column

\framebreak % End of the first column

%----------------------------------------------------------------------------------------
%	HEADING
%----------------------------------------------------------------------------------------

\cvheading{Samuel Coelho} % Large heading - your name

\cvsubheading{Software Engineer} % Subheading - your occupation/specialization

%----------------------------------------------------------------------------------------
%	ABOUT ME
%----------------------------------------------------------------------------------------

\aboutme{About Me}{
	About me
}

%----------------------------------------------------------------------------------------
%	EDUCATION
%----------------------------------------------------------------------------------------

\CVSection{Education}

%------------------------------------------------

\CVItem{
	2010 - 2015, Instituto Superior Técnico, Lisbon
}
{MEng in Computer Science}

%------------------------------------------------

\Sep % Extra whitespace after the end of a major section

%----------------------------------------------------------------------------------------
%	EXPERIENCE
%----------------------------------------------------------------------------------------

\CVSection{Experience}

%------------------------------------------------

\CVItem{October 2018 - present, \textit{Software Engineer}, Practio}{
	Full stack development of a platform used in the Healthcare industry.
	Platform to help healthcare users, such as, pharmacists 
	and doctors to register information about patients and 
	help them perform severals types of consultations, 
	such as, travel vaccinations.
	
	Tech stack:
\begin{itemize}
	\item Backend: NodeJS, ExpressJS, MongoDB
\end{itemize}
\begin{itemize}
	\item Frontend: Javascript (ES6), React, Redux, Stylus
\end{itemize}
}

%------------------------------------------------

\CVItem{June 2017 - October 2018, \textit{Android Developer}, Tekever}{
	Worked as an Android native developer in several apps.
	
	Native Android apps for some of the biggest Portuguese 
	electrical power providers. 
	The main goal was to help workers perform their job of 
	installing new equipment and getting the values 
	from smart meters.

	Tech stack:
\begin{itemize}
	\item Java
\end{itemize}
}

%------------------------------------------------

\CVItem{November 2015 - June 2017, \textit{Software Developer}, Novabase}{
Worked on Android, iOS and web applications for several clients.

Native android apps for two clients.\\
Tech stack:
\begin{itemize}
	\item Java
\end{itemize}

Native iOS app for one client.\\
Tech stack:
\begin{itemize}
	\item ObjectiveC
\end{itemize}

Web application for one client.\\
Tech stack:
\begin{itemize}
	\item Backend: Java, OracleDB
	\item Frontend: Javascript, CSS, HTML
\end{itemize}
}

%------------------------------------------------

\CVItem{October 2013 - October 2015, \textit{Software Developer}, FenixEdu}{
Organization inside college that was in charge of developing 
the academic information system, 
used there and in other institutions, to manage everything
related to the students, staff, etc. 
Other projects were also being developed here. 
As a developer here (at the same time I was graduating) 
I worked in python libraries and several web applications.

Python libraries. The main goal of these libraries was to provide
functions for other apps to use the FenixAPI features, such as, get
students data, enroll student in exams and much more

Tech stack:
\begin{itemize}
	\item Python, Django
\end{itemize}

Web applications. Includes the academic information system
and other side projects inside the organization.

Tech stack:
\begin{itemize}
	\item Backend: Java, Struts, Jersey, Python, 

	Django, FenixFramework (ORM developed inhouse)
	\item Frontend: Javascript, HTML, 
	
	Javascript, Angular, React
\end{itemize}
}

\CVItem{October 2013 - October 2015, \textit{Software Developer}, FenixEdu}{
Organization inside college that was in charge of developing 
the academic information system, 
used there and in other institutions, to manage everything
related to the students, staff, etc. 
Other projects were also being developed here. 
As a developer here (at the same time I was graduating) 
I worked in python libraries and several web applications.

Python libraries. The main goal of these libraries was to provide
functions for other apps to use the FenixAPI features, such as, get
students data, enroll student in exams and much more

Tech stack:
\begin{itemize}
	\item Python, Django
\end{itemize}

Web applications. Includes the academic information system
and other side projects inside the organization.

Tech stack:
\begin{itemize}
	\item Backend: Java, Struts, Jersey, Python, 

	Django, FenixFramework (ORM developed inhouse)
	\item Frontend: Javascript, HTML, 
	
	Javascript, Angular, React
\end{itemize}
}

%------------------------------------------------

% \Sep % Extra whitespace after the end of a major section

%----------------------------------------------------------------------------------------
%	COMMUNICATION SKILLS
%----------------------------------------------------------------------------------------

% \CVSection{Communication Skills}

%------------------------------------------------

% \CVItem{2015, \textit{Oral Presentation}, California Business Conference}{Presented research I conducted for my Masters of Engineering degree.}

%------------------------------------------------

% \CVItem{2014, \textit{Poster}, Annual Business Conference (Oregon)}{As part of the course work for BUS320, I created a poster analyzing several local businesses and presented this at a conference.}

%------------------------------------------------

\Sep % Extra whitespace after the end of a major section

%----------------------------------------------------------------------------------------
%	SKILLS
%----------------------------------------------------------------------------------------

\CVSection{Software Development Skills}

%------------------------------------------------

\CVItem{Programming}
{\begin{tabular}{p{0.2\textwidth} p{0.2\textwidth} p{0.2\textwidth}}
\bluebullet Javascript & 
\bluebullet React\\ 
\bluebullet Redux\\
\bluebullet Redux Sagas\\
\bluebullet CSS\\
\bluebullet Stylus\\
\bluebullet MongoDB\\
\bluebullet NodeJS &
\end{tabular}}

%------------------------------------------------

\Sep % Extra whitespace after the end of a major section

%----------------------------------------------------------------------------------------
%	NEW PAGE DELIMITER
%	Place this block wherever you would like the content of your CV to go onto the next page
%----------------------------------------------------------------------------------------

\clearpage % Start a new page

\userinformation % Print your information in the left column

\framebreak % End of the first column

%----------------------------------------------------------------------------------------
%	AWARDS
%----------------------------------------------------------------------------------------

\CVSection{Awards}

%------------------------------------------------

\CVItem{2010, \textit{Postgraduate Scholarship}, Cornell University}{Awarded to the top student in their final year of a Bachelors degree.}

%------------------------------------------------

\Sep % Extra whitespace after the end of a major section

%----------------------------------------------------------------------------------------
%	INTERESTS
%----------------------------------------------------------------------------------------

\CVSection{Interests}

%------------------------------------------------

\CVItem{Professional}{Data analysis, company profiling, risk analysis, economics, web design, web app creation, software design, marketing}

%------------------------------------------------

\CVItem{Personal}{Piano, chess, cooking, dancing, running}

%------------------------------------------------

\Sep % Extra whitespace after the end of a major section

%----------------------------------------------------------------------------------------

\end{document}
